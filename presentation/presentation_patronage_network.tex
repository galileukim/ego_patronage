% Options for packages loaded elsewhere
\PassOptionsToPackage{unicode}{hyperref}
\PassOptionsToPackage{hyphens}{url}
%
\documentclass[
  ignorenonframetext,
]{beamer}
\usepackage{pgfpages}
\setbeamertemplate{caption}[numbered]
\setbeamertemplate{caption label separator}{: }
\setbeamercolor{caption name}{fg=normal text.fg}
\beamertemplatenavigationsymbolsempty
% Prevent slide breaks in the middle of a paragraph
\widowpenalties 1 10000
\raggedbottom
\setbeamertemplate{part page}{
  \centering
  \begin{beamercolorbox}[sep=16pt,center]{part title}
    \usebeamerfont{part title}\insertpart\par
  \end{beamercolorbox}
}
\setbeamertemplate{section page}{
  \centering
  \begin{beamercolorbox}[sep=12pt,center]{part title}
    \usebeamerfont{section title}\insertsection\par
  \end{beamercolorbox}
}
\setbeamertemplate{subsection page}{
  \centering
  \begin{beamercolorbox}[sep=8pt,center]{part title}
    \usebeamerfont{subsection title}\insertsubsection\par
  \end{beamercolorbox}
}
\AtBeginPart{
  \frame{\partpage}
}
\AtBeginSection{
  \ifbibliography
  \else
    \frame{\sectionpage}
  \fi
}
\AtBeginSubsection{
  \frame{\subsectionpage}
}
\usepackage{lmodern}
\usepackage{amssymb,amsmath}
\usepackage{ifxetex,ifluatex}
\ifnum 0\ifxetex 1\fi\ifluatex 1\fi=0 % if pdftex
  \usepackage[T1]{fontenc}
  \usepackage[utf8]{inputenc}
  \usepackage{textcomp} % provide euro and other symbols
\else % if luatex or xetex
  \usepackage{unicode-math}
  \defaultfontfeatures{Scale=MatchLowercase}
  \defaultfontfeatures[\rmfamily]{Ligatures=TeX,Scale=1}
\fi
% Use upquote if available, for straight quotes in verbatim environments
\IfFileExists{upquote.sty}{\usepackage{upquote}}{}
\IfFileExists{microtype.sty}{% use microtype if available
  \usepackage[]{microtype}
  \UseMicrotypeSet[protrusion]{basicmath} % disable protrusion for tt fonts
}{}
\makeatletter
\@ifundefined{KOMAClassName}{% if non-KOMA class
  \IfFileExists{parskip.sty}{%
    \usepackage{parskip}
  }{% else
    \setlength{\parindent}{0pt}
    \setlength{\parskip}{6pt plus 2pt minus 1pt}}
}{% if KOMA class
  \KOMAoptions{parskip=half}}
\makeatother
\usepackage{xcolor}
\IfFileExists{xurl.sty}{\usepackage{xurl}}{} % add URL line breaks if available
\IfFileExists{bookmark.sty}{\usepackage{bookmark}}{\usepackage{hyperref}}
\hypersetup{
  pdftitle={Government through Patronage},
  pdfauthor={Galileu Kim},
  hidelinks,
  pdfcreator={LaTeX via pandoc}}
\urlstyle{same} % disable monospaced font for URLs
\newif\ifbibliography
\setlength{\emergencystretch}{3em} % prevent overfull lines
\providecommand{\tightlist}{%
  \setlength{\itemsep}{0pt}\setlength{\parskip}{0pt}}
\setcounter{secnumdepth}{-\maxdimen} % remove section numbering
\usetheme{metropolis}
\setbeamertemplate{footline}[page number]{}

% packages
\usepackage{xcolor}
\usepackage{ragged2e}
\usepackage{setspace}
\usepackage{etoolbox}
\usepackage{caption}
\usepackage{graphicx}
\usepackage{float}
\usepackage{wrapfig}
\usepackage{blindtext}
\captionsetup[figure]{font=scriptsize} % set size of caption
\AtBeginEnvironment{knitrout}{\singlespacing}
\useoutertheme{infolines} % Alternatively: miniframes, infolines, split
\useinnertheme{circles}
\date[]{\today}

% colors
\definecolor{princeton}{rgb}{1.0, 0.6, 0.2} % princeton (primary)
\definecolor{arsenic}{rgb}{0.23, 0.27, 0.29}

\setbeamercolor{title}{fg=black}
\setbeamercolor{frametitle}{fg=white,bg=arsenic}
\setbeamercolor{section in head/foot}{fg=arsenic,bg=princeton}
\setbeamerfont{frametitle}{size=\Large}
\setbeamercolor{palette primary}{fg=white,bg=arsenic}
\setbeamercolor{palette secondary}{fg=arsenic,bg=princeton}
\setbeamercolor{palette tertiary}{fg=princeton,bg=arsenic}
\setbeamercolor{palette quaternary}{fg=princeton,bg=arsenic}
\setbeamercolor{structure}{fg=princeton} % itemize, enumerate, etc
\setbeamercolor{section in toc}{fg=arsenic} % TOC sections
\setbeamerfont{subtitle}{size=\small}

% override palette coloring with secondary
\setbeamercolor{subsection in head/foot}{bg=arsenic,fg=white}

% defining font
\usepackage[T1]{fontenc}

% Cabin
% \usepackage[sfdefault,condensed]{cabin}

% Helvetica
\usepackage[scaled]{helvet}
\renewcommand\familydefault{\sfdefault}

% Inria Sans
%\usepackage[lining]{InriaSans}
%\renewcommand*\oldstylenums[1]{{\fontfamily{InriaSans-OsF}\selectfont #1}}
%% The font package uses mweights.sty which has som issues with the
%% \normalfont command. The following two lines fixes this issue.
%\let\oldnormalfont\normalfont
%\def\normalfont{\oldnormalfont\mdseries}

% FiraSans
%\usepackage[sfdefault]{FiraSans}
%\renewcommand*\oldstylenums[1]{{\firaoldstyle #1}}
%\usepackage{ragged2e}

% aesthetics
\setbeamersize{text margin left=10mm,text margin right=10mm}

% source command
\newcommand{\source}{\footnotesize\textcolor{black!25}}

% suppress section slides
\AtBeginPart{}
\AtBeginSubsection{}
\AtBeginSection{}
\setlength{\emergencystretch}{0em}
\setlength{\parskip}{0pt}

\title{Government through Patronage}
\subtitle{Bargaining for Education in Decentralized Brazil}
\author{Galileu Kim}
\date{}
\institute{Princeton University}

\begin{document}
\frame{\titlepage}

\hypertarget{introduction}{%
\section{Introduction}\label{introduction}}

\begin{frame}{Introduction}
\begin{itemize}
\tightlist
\item
  Why should we care about patronage in specific?
\item
  What can I bring into the table that scholars have not thought about?
\item
  What are the gaps in the literature and the theoretical and empirical
  contribution I bring to the table?
\item
  Can I dream of being ambitious? Yes, I am much better now than I used
  to be before.
\item
  Bring both my theoretical treatment and empirical estimation to the
  fore.
\end{itemize}
\end{frame}

\hypertarget{different-types-of-patronage}{%
\section{Different types of
patronage}\label{different-types-of-patronage}}

\begin{frame}{Different types of patronage}
\begin{itemize}
\tightlist
\item
  Politicians value different types of positions for patronage.
\item
  For the mayor, who relies less on particularistic transfers (more on
  that later), control over the higher echelons of the bureaucracy is
  more important.
\item
  For city councilors, ties to the local community are more important:
  neighbors rely on public sector employment.
\end{itemize}
\end{frame}

\hypertarget{betting-on-horses}{%
\section{Betting on horses}\label{betting-on-horses}}

\begin{frame}{Betting on horses}
\begin{itemize}
\tightlist
\item
  Voters are strategic.
\item
  They should contribute to candidates who they believe will make them
  likely to obtain employment.
\item
  If I can link the people who donated tho candidates with the
  likelihood of them entering a bureaucracy.
\item
  Something very interesting can be done.
\end{itemize}
\end{frame}

\hypertarget{differentiating-types-of-donors}{%
\section{Differentiating types of
donors}\label{differentiating-types-of-donors}}

\begin{frame}{Differentiating types of donors}
\begin{itemize}
\tightlist
\item
  For some voters, campaign contributino is symbolic.
\item
  In Brazil, it is more likely that donors are thinking strategically
  about their candidates.
\item
  Who do they contribute to? What is there to win?
\item
  Who are the public sector workers more likely to donate?
\item
  How does this affect their likelihood of being in office?
\end{itemize}
\end{frame}

\hypertarget{campaign-contribution-and-employment}{%
\section{Campaign contribution and
employment}\label{campaign-contribution-and-employment}}

\begin{frame}{Campaign contribution and employment}
\begin{itemize}
\tightlist
\item
  It is clear that this study dialogues with another set of studies.
\item
  Colonnelli et al.~come to mind. The problem is that they have already
  done it.
\item
  Well, at least for mayors. Look at how campaign contributions increase
  the likelihood of obtaining employment.
\item
  So what else is there to be done?
\item
  Emphasizing the city councilor link is not enough. I need to do more.
\end{itemize}
\end{frame}

\hypertarget{trade-offs}{%
\section{Trade-offs}\label{trade-offs}}

\begin{frame}{Trade-offs}
\begin{itemize}
\tightlist
\item
  For voters t, there is a clear trade-off between the immediate gains
  of employment and the uncertain gains from public goods provision.
\item
  It is essentially the fact of hedging and how citizens are willing to
  take the public sector employment for themselves, which can help them
  provide for their family needs, rather than having to rely on public
  schools or well-staffed hospitals.
\item
  It is a tragedy of the commons, but more subtle.
\item
  Let's spell this out.
\end{itemize}
\end{frame}

\hypertarget{context-and-data}{%
\section{Context and data:}\label{context-and-data}}

\begin{frame}{Context and data:}
\begin{itemize}
\tightlist
\item
  Brazil has multiple city councilors, who rely on patronage networks to
  be reelected.
\item
  The question then is what is the additional
\end{itemize}
\end{frame}

\end{document}
