\documentclass[12pt,a4paper]{article}
\usepackage{setspace} \onehalfspacing
\usepackage[top = 1in, bottom = 1in, left = 1in, right = 1in]{geometry}
\usepackage[utf8]{inputenc}
\usepackage{amsmath}
\usepackage{amsfonts}
\usepackage{amssymb}
\usepackage{amsthm}
\usepackage{graphicx}
\usepackage{natbib}
\usepackage{caption}
\usepackage{subcaption}
\usepackage{float}
\usepackage{pdflscape}
\usepackage{booktabs}
\usepackage{dcolumn}
\usepackage{pdflscape}
% \usepackage{hyperref}
\usepackage[hyphens]{url}
\usepackage{enumitem}
\usepackage[table]{xcolor}
\usepackage{authblk}
\usepackage{appendix}
\usepackage{titletoc}
\usepackage{fancyhdr}
\usepackage{hyperref}

\bibliographystyle{apsr}

\newcommand{\note}{\textcolor{blue}}

\title{A partisan affair: Mapping patronage in municipal bureaucracies of Brazil}
\author{Galileu Kim}
\affil{Princeton University}

\begin{document}

\maketitle

\abstract{How extensive is patronage in Brazil? What are the differences in observables between partisan affiliates and non-partisan members? Leveraging a novel dataset of partisan affiliation and employment data on every municipal bureaucrat in Brazil, I find that party members are more likely to be overcompensated than their peers, concentrating in areas of executive leadership, while being less educated than their peers. In addition, their employment spells tend to be more durable over time, leading to negative consequences. These findings provide raise important questions regarding the duration of partisan ties and its consequences for local bureaucracies.}

\newpage

\section{Introduction}
\label{sec:intro}

How does 

\section{Context and Data}
\label{sec:context}

\section{Empirical strategy}
\label{sec:empirical}

\section{Conclusion}
\label{sec:conclusion}

\newpage

\bibliography{patronage}

\end{document}