\documentclass[12pt,a4paper]{article}
\usepackage{setspace} \onehalfspacing
\usepackage[top = 1in, bottom = 1in, left = 1in, right = 1in]{geometry}
\usepackage[utf8]{inputenc}
\usepackage{amsmath}
\usepackage{amsfonts}
\usepackage{amssymb}
\usepackage{amsthm}
\usepackage{graphicx}
\usepackage{natbib}
\usepackage{caption}
\usepackage{subcaption}
\usepackage{float}
\usepackage{pdflscape}
\usepackage{booktabs}
\usepackage{dcolumn}
\usepackage{pdflscape}
% \usepackage{hyperref}
\usepackage[hyphens]{url}
\usepackage{enumitem}
\usepackage[table]{xcolor}
\usepackage{authblk}
\usepackage{appendix}
\usepackage{titletoc}
\usepackage{fancyhdr}
\usepackage{hyperref}

\bibliographystyle{apsr}

\newcommand{\note}{\textcolor{blue}}

\title{A partisan affair: Mapping patronage in municipal bureaucracies of Brazil}
\author{Galileu Kim}
\affil{Princeton University}

\begin{document}

\maketitle

\abstract{How extensive is patronage in Brazil? What are the differences in observables between partisan affiliates and non-partisan members? Leveraging a novel dataset of partisan affiliation and employment data on every municipal bureaucrat in Brazil, I find that party members are more likely to be overcompensated than their peers, concentrating in areas of executive leadership, while being less educated than their peers. In addition, their employment spells tend to be more durable over time, leading to the build-up of party members in the bureaucracy over time. These findings provide raise important questions regarding the nature of patronage, party building and its consequences for local bureaucracies.}

\newpage

\section{Introduction}
\label{sec:intro}

Who benefits from patronage? And what are its consequences for local administration? 

Literature on patronage and selection: \citet{robinson2013political}. Extant literature on selection of public officials have primarily focused on who becomes a politician \citet{dal2017becomes}. There remains unanswered questions regarding who enters the bureaucracy, what types of employment they held, and what are the differential compensations among public officials.

Empirical research on allocation of public sector jobs has consistently found that politically motivated allocation of public sector jobs is an important determinant of public employment in the developed and developing world \citep{finan2017personnel}. In the United States, appointments to the federal bureaucracy involve considerations of party loyalty and ideological alignment with the president \citep{lewis2010politics, hollibaugh2014presidents}. In Brazil, whether it be at the presidential \citep{pracca2011political} or at the municipal level \citep{colonnelli2018patronage,brollo2017victor}, politicians have discretion over bureaucratic appointments and frequently use these to further their political goals.

A set of explanations have been proposed to explain the determinants of this allocation, whether it be to reduce frictions in policy implementation due to ideological divergence \citep{krause2016experiential}, exploit the benefits of strong ties \citep{toral2019benefits} or to reward followers for campaign contributions \citep{colonnelli2018patronage}. These findings have provided important insights into the drivers of patronage appointments into employment in the public sector. Yet as important as the drivers of public sector appointment are the characteristics of those entering the bureaucracy \citep{finan2015personnel}. Who becomes a bureaucrat? Are political appointees systematically different from their non-partisan counterparts? And where, within the bureaucracy, do these patronage appointees go?

In this paper, I focus on party membership as a determinant of not only whether or not individuals enter into public sector employment, but the kind of employment you receive. In particular, using a rich panel dataset of all public sector workers, I identify the party membership of all municipal bureaucrats in Brazil to provide a novel set of empirical findings: 1) the pre-bureaucracy characteristics of public sector workers, 2) 

In this paper, I leverage micro-level partisan information with employment data to assess the degree to which there are differential compensation structures and differences in observables among partisan and non-partisan members of the bureaucracy. The main finding is that patronage is an elite game, benefitting the wealthiest formal sector workers and allocating them into the best-paying jobs in the public sector. The benefits of public sector employment accrue to a wealthier class of workers, in particular those already in profitable positions in the private sector. Additionally, wealthy party members accumulate jobs at higher levels of government such as executive leadership and administration, which are better compensated. In contrast, less profitable offices in frontline services experience less exposure to partisan members.

Additionally, party members are concentrated in tenured positions. Over X\% of career bureaucrats are affiliated to a party. Party members tend to stay in the bureaucracy for longer, and receive a higher compensation than their non-partisan counterparts. 

Patronage in the allocation of public sector jobs 

\section{Context and Data}
\label{sec:context}



\section{Empirical strategy}
\label{sec:empirical}

\section{Conclusion}
\label{sec:conclusion}

\newpage

\bibliography{patronage}

\end{document}