\documentclass[12pt,a4paper]{article}
\usepackage{setspace} \onehalfspacing
\usepackage[top = 1in, bottom = 1in, left = 1in, right = 1in]{geometry}
\usepackage[utf8]{inputenc}
\usepackage{amsmath}
\usepackage{amsfonts}
\usepackage{amssymb}
\usepackage{amsthm}
\usepackage{graphicx}
\usepackage{natbib}
\usepackage{caption}
\usepackage{subcaption}
\usepackage{float}
\usepackage{pdflscape}
\usepackage{booktabs}
\usepackage{dcolumn}
\usepackage{pdflscape}
% \usepackage{hyperref}
\usepackage[hyphens]{url}
\usepackage{enumitem}
\usepackage[table]{xcolor}
\usepackage{authblk}
\usepackage{appendix}
\usepackage{titletoc}
\usepackage{fancyhdr}
\usepackage{hyperref}

\bibliographystyle{apsr}

\newcommand{\note}{\textcolor{blue}}

\title{A partisan affair: Mapping patronage in municipal bureaucracies of Brazil}
\author{Galileu Kim}
\affil{Princeton University}

\begin{document}

\maketitle

\abstract{How extensive is patronage in Brazil? What are the differences in observables between partisan affiliates and non-partisan members? Leveraging a novel dataset of partisan affiliation and employment data on every municipal bureaucrat in Brazil, I find that party members are more likely to be overcompensated than their peers, concentrating in areas of executive leadership, while being less educated than their peers. In addition, their employment spells tend to be more durable over time, leading to the build-up of party members in the bureaucracy over time. These findings provide raise important questions regarding the nature of patronage, party building and its consequences for local bureaucracies.}

\newpage

\section{Introduction}
\label{sec:intro}

Who benefits from patronage? And what are its consequences for local administration? 

Literature on patronage and selection: \citet{robinson2013political}. Extant literature on selection of public officials have primarily focused on who becomes a politician \citet{dal2017becomes}. There remains unanswered questions regarding who enters the bureaucracy, what types of employment they held, and what are the differential compensations among public officials.

Empirical research on allocation of public sector jobs has consistently found that politically motivated allocation of public sector jobs is an important determinant of public employment in the developed and developing world \citep{finan2017personnel}. In the United States, appointments to the federal bureaucracy involve considerations of party loyalty and ideological alignment with the president \citep{lewis2010politics, hollibaugh2014presidents}. In Brazil, whether it be at the presidential \citep{pracca2011political} or at the municipal level \citep{colonnelli2018patronage,brollo2017victor}, politicians have discretion over bureaucratic appointments and frequently use these to further their political goals.

A set of explanations have been proposed to explain the determinants of this allocation, whether it be to reduce frictions in policy implementation due to ideological divergence \citep{krause2016experiential}, exploit the benefits of strong ties \citep{toral2019benefits} or to reward followers for campaign contributions \citep{colonnelli2018patronage}. These findings have provided important insights into the drivers of patronage appointments into employment in the public sector. Yet as important as the drivers of public sector appointment are the characteristics of those entering the bureaucracy \citep{finan2015personnel}. Who becomes a bureaucrat? Are political appointees systematically different from their non-partisan counterparts? And where, within the bureaucracy, do these patronage appointees go?

In this paper, I focus on party membership as a determinant of not only whether or not individuals enter into public sector employment, but the type of employment they receive. In particular, using a rich panel data set of all public sector workers, I identify the party membership of all municipal bureaucrats in Brazil to provide a novel set of empirical findings: 1) the pre-bureaucracy characteristics of public sector workers, 2) employment trajectories within the bureaucracy and 3) post-public sector employment of party members. This complete revolving-door of party and non-party members provide a unique frame-by-frame evolution of the employment trajectories of pre- and post-bureaucrats in a developing world context.

The main finding is that patronage primarily benefits a local economic elite, accruing to the richest formal sector workers and allocating them into the best-paying jobs in the municipal bureaucracy. These higher compensation structures are not commensurate to skills on a set of observable qualifications, such as education level and work experience, suggesting that these benefits accrue from channels other than individual skills. Additionally, party members accumulate jobs at higher levels of government such as executive leadership and administration, which are better compensated, as well as benefiting from longer income streams due to privileged access to tenured contracts.

These findings suggest that, overall, patronage does not necessarily accrue to poor voters, contrary to theoretical and empirical findings in studies of clientelism in the developing world \citep{stokes2013brokers}. A different logic seems to be at play. Patronage to party loyalists can be used strategically as a mean of securing control over the bureaucracy -- through tenured contracts -- as well as securing buy-in from wealthy patrons or notables in the local economy, consistent with empirical findings by \citet{colonnelli2018patronage}. Patronage therefore can serve a crucial role in securing access to economic resources through wealthy patrons, which in turn can be allocated to finance campaigns and other exercises in party building.

This paper contributes to literature on clientelism that outlines the political logic of patronage allocations. The contribution is twofold: 1) first, I find that patronage is a patron-elite game, generating a form of patronage that is distinct from the politician-voter nexus that has been traditionally the focus of extant literature on clientelism \citet{stokes2013brokers, diaz2016political} and 2) I find that the benefits of patronage go beyond simply an electoral pay-off. Instead, what patronage accomplishes is securing access to economic resources that can be tapped into once employment is offered to wealthy patrons, similar in spirit to the theoretical findings by \citet{robinson2013political}. Patronage binds parties and patrons, in effect capturing the benefits of public sector employment to finance efforts towards party-building.

The paper is structured as follows. Section \ref{sec:context} provides context for the data and the hiring process for local bureaucracies, while section \ref{sec:descriptive} provides some descirptive statistics outlining the differences between partisan and non-partisan members. Section \ref{sec:empirical} outlines the empirical strategy. Section \ref{sec:conclusion} concludes.

\section{Context and Data}
\label{sec:context}

\subsection{Partisan affiliation in Brazil}

Partisanship in Brazil is voluntary and widespread, with over 11 percent of registered voters affiliated to a party \citep{speck2015estudo}\footnote{For context, in most OECD countries party registration does not exceed 5 percent of the electorate. See \citet{biezen2014decline}.}. Registration occurs in the following sequence: a voter reports to a local party office, and the party officials then register the voter officially through the \emph{Tribunal Regional Eleitoral} (Regional Electoral Office). This registration is then centralized by the \emph{Tribunal Superior Eleitoral} (Supreme Electoral Office) and updated accordingly. If there are overlapping registrations, former ones become annulled and are reported as irregular to local party officials. Each voter is therefore only allowed to register for a single party, without any ceilings or floors regarding the duration of this affiliation.

There is an ongoing debate on the strength of partisan ties in Brazil. On the one hand, scholars have noted that partisanship in Brazil is weak, meaning that politicians and voters do not have strong party loyalty and often ``switch'' to other parties \citet{desposato2006parties,ames2002deadlock}. On the other, some scholars have noted that party ties have grown in strength over time, in particular for leftist programmatic parties such as the PT \citep{samuels2014power,samuels2006sources}. Part of this debate owes to disagreements on how to measure party strength, whether it be testing voters' prior knowledge of party's ideological positions or if instead, it should be measured by testing whether voters issue ballots for individual candidates or their party labels -- with each one of these measures painting an opposite picture of the relative strength of party ties.

One thing is clear: party ties at the electoral level are durable, with most voters remaining affiliated to a single party for their entire life, as highlighted by figure \emph{FIGURE}. Noting that party affiliations are registered at the municipal level, this empirical fact aligns with qualitative evidence provided by \citet{palmeira1995comicios}, who notes that parties at the local level constitute \emph{grupos pol\'{i}ticos} (political factions) with well-defined boundaries and power disputes. This relative stability of party ties at the electorate level for the minority of voters who are registered with a parties suggests a distinct dynamic tying an elite group of party members to the city hall.

\subsection{Municipal employment and patronage in Brazil}

\section{Descriptive Statistics}
\label{sec:descriptive}

\section{Empirical strategy}
\label{sec:empirical}

\section{Conclusion}
\label{sec:conclusion}

\newpage

\bibliography{patronage}

\end{document}